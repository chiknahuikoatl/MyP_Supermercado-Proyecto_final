\documentclass[a4paper,12pt]{report}

\usepackage[spanish]{babel}
\usepackage[utf8]{inputenc}
\usepackage{amsmath}
\usepackage{fancyhdr}
\usepackage{graphicx}
\graphicspath{ {imagenes/} }

\usepackage{hyperref}
\hypersetup{
    colorlinks=true,
    linkcolor=blue,
    filecolor=magenta,
    urlcolor=cyan,
}

\title{\bf Proyecto Final}
\author{Jerónimo Almeida Rodríguez}
\author{Martin Felipe Espinal Cruces}
\date{\today}

\pagestyle{fancy}
\lhead{Almeida \& Espinal}
\chead{Proyecto Final}
\rhead{\today}
\lfoot{jalrod@ciencias.unam.mx}
\rfoot{cofy43@ciencias.unam.mx}

\begin{document}
\begin{titlepage}
    \centering
    {\scshape\Huge Universidad Nacional Autónoma de México \par}
    \vspace{2cm}
    {\scshape\huge Modelado y Programación\par}
    \vspace{2cm}
    {\huge\bfseries Proyecto Final\par}
    \vspace{1.5cm}
    {\Large\textsc Jerónimo Almeida Rodríguez \par}
    \vspace{.25cm}
    {\large\texttt{ jalrod@ciencias.unam.mx}\par}
    \vspace{1cm}
    {\Large\textsc Martin Felipe Espinal Cruces \par}
    \vspace{.25cm}
    {\large\texttt{cofy43b@ciencias.unam.mx}\par}
    \vspace{2cm}
    \vfill
    \begin{figure}[hb!]
        \includegraphics[width=.3\textwidth]
            {../../logos/escudo_f-ciencias.png}\hfill
        \includegraphics[width=.3\textwidth]
            {../../logos/Escudo_UNAM.png}\hfill
    \end{figure}
\end{titlepage}

\section*{Planteamiento del Problema.}{ El planteamiento del problema se encuentra
en el pdf del proyecto.\\}
\section*{Objetivo.}{Simular un supermercado, aplicando los conocimientos obtenidos
en clase sobre concurrencia y otros temas.}
\section*{Desarrollo.}{El programa consiste de tres partes principales:
\begin{itemize}
    \item{}
\end{itemize}}
\section*{Solución.}{}

\begin{thebibliography}{}
\end{thebibliography}
\end{document}
